\documentclass{article}

\usepackage[ngerman]{babel}
\usepackage[utf8]{inputenc}
\usepackage[T1]{fontenc}
\usepackage{hyperref}
\usepackage{csquotes}

\usepackage[
    backend=biber,
    style=apa,
    sortlocale=de_DE,
    natbib=true,
    url=false,
    doi=false,
    sortcites=true,
    sorting=nyt,
    isbn=false,
    hyperref=true,
    backref=false,
    giveninits=false,
    eprint=false]{biblatex}
\addbibresource{../references/bibliography.bib}

\title{Notizen zum Projekt Data Ethics}
\author{Marina Brun}
\date{\today}

\begin{document}
\maketitle



\tableofcontents

\section{Fragestellung}
Welche ethischen Bedenken gibt es hinsichtlich des Datenschutzes und der Privatsphäre in KI-Anwendung?

\section{Einleitung}

Künstliche Intelligenz bezieht sich auf die Entwicklung von Maschinen, damit diese Aufgaben ausführen können, für die normalerweise menschliche Intelligenz erforderlich ist. Dazu gehören Techniken wie maschinelles Lernen, neuronale Netzwerke und Datenanalyse. Maschinelles Lernen ermöglicht es Computern, aus Erfahrungen zu lernen und sich selbst zu verbessern. Neuronale Netzwerke sind ein wichtiger Bestandteil von KI-Systemen und helfen bei der Verarbeitung von Informationen ähnlich wie das menschliche Gehirn.

\section{Ki Training}

Man trainiert KI-Systeme durch den Prozess des Machine Learning Trainings, bei dem das System anhand von bereitgestellten Daten lernt, Muster zu erkennen, Prognosen zu erstellen oder Entscheidungen zu treffen. Es gibt zwei Hauptmethoden für das KI-Training: überwachtes Lernen, das gelabelte Eingabe- und Ausgabedaten benötigt, und nicht überwachtes Lernen, das dies nicht erfordert. Das Training ist entscheidend für die Entwicklung von KI-Systemen, die spezifische Aufgaben ausführen können.

\section{Datenschutzbedenken}
 
Die Bedenken beim Datenschutz in KI beziehen sich auf Fragen und Risiken hinsichtlich Datenschutz, Privatsphäre, und Ethik. Es ist wichtig sicherzustellen, dass KI-Systeme so konzipiert, entwickelt und eingesetzt werden, dass die Rechte und Interessen des Einzelnen und der Gesellschaft gewahrt bleiben. Es muss ein stärkeres Bewusstsein für Datenschutz und die Einhaltung von Datenschutzgesetzen geschaffen werden, um sicherzustellen, dass die Einwilligung für KI-Training eingeholt wird.

\printbibliography

\end{document}
