\documentclass{article}


\usepackage[ngerman]{babel}
\usepackage[utf8]{inputenc}
\usepackage[T1]{fontenc}
\usepackage{hyperref}
\usepackage{csquotes}
\usepackage[a4paper]{geometry}
\usepackage{graphicx}

\usepackage[
    backend=biber,
    style=apa,
    sortlocale=de_DE,
    natbib=true,
    url=false,
    doi=false,
    sortcites=true,
    sorting=nyt,
    isbn=false,
    hyperref=true,
    backref=false,
    giveninits=false,
    eprint=false]{biblatex}
\addbibresource{../references/bibliography.bib}

\title{Review des Papers "Ethische Probleme bei der nutzung von KI" }
\author{Marina Brun}
\date{\today}

\begin{document}
\maketitle







\section{Künstlche Intelligenz}

Du hast gut erklärt was eine KI überhaupt ist, sodass auch Menschen ohne Vorkentnisse grob verstehen um was es geht. Mit den Beispielen wurde auch klarer um was es sich bei der künstlichen Intelligenz eigentlich handelt. KI ist aber nicht ganz eine Maschine, sonder die Fähigkeit einer Maschine mit der Intelligenz eines Menschen. 

\vspace{2mm} Die Idee, Beispiele zu nennen, wo die KI in unserem Alltag vorkommt, hilft den Lesern zu verstehen mit was sie sich beschäftigen wenn sie diesen Text lesen. 

\vspace{2mm} Du erläuterst dass die KI trainiert wird indem man sie mit Daten füttert. Diese Beschreibung ist verständlich und genau beschrieben. Jedoch werden die einzelnen Schritte nich genauer in Betracht gezogen. Eine detaillierte Erklärung zu den technsichen Details wäre hilfreich. Man könnte zum Beispeil Begriffe wie "neuronale Netzwerke" oder "maschinelles Lernen" erläutern, damit das Verständnis vertieft wird. Du gehst darauf ein, dass man die Daten kategorisieren und labeln muss.


\section{Fragestellung}

Das Dokument ist gut struktuirert mit Abschnitten und Titeln, sodass ein klarer Überblick für den Leser und ein verständlicher Aufbau entsteht.

\vspace{2mm} Durch das, dass du Begriffe wie Datenschutz gut erklärt und als Argument präsentiert hast, ist der Text gut zu verstehen und auch nachzuvollziehen.

\vspace{2mm} Die Erklärung von den potenziellen Gefahren ist sehr präzise und ebenfalls durch das Erläutern nachvollziehbar. Auf das Thema Datenschutz gehst du kritisch ein, was einen guten Einblick auf die Gefahren gibt, die es geben kann wenn die Richtlinien nicht eingehalten werden. 

\vspace{2mm} Du machst klare Beispiele welche zum Verständnis des Lesers beitragen, wie zum Beipsiel der Identitätsdiebstahl. Du hebst hervor, dass immer eine Gefahr besteht wenn man der KI seine Daten gibt, denn diese können missbraucht werden. Dies löst beim Leser die Warnung auf immer aufzupassen welche Daten man der KI anvertraut.


\section{Schlusswort}

Du hast das Thema übersichtlich zusammengefasst. Dem Leser ist jetzt klar, es nicht nur positive sondern auch negative Aspekte bei der Nutzung von künstlicher Intelligenz gibt. Die Fragestellung wurde schlussendlich ausführlich und komplett beantwortet. Du erklärst welche Massnahmen zu beachten sind, für eine sichere Nutzung der KI. 

\vspace{2mm} Durch die Nutzung von guten Quellen, weiss man dass die Argumente glaubwürdig sind und es zeigt auch, dass du dich gut informiert hast.

\printbibliography

\end{document}

